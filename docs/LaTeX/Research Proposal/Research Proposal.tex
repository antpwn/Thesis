\documentclass[10pt,a4paper]{article}
\usepackage[utf8x]{inputenc}
\usepackage{ucs}
\usepackage[hebrew,english]{babel}
\usepackage{amsmath}
\usepackage{amsfonts}
\usepackage{amssymb}
\usepackage{graphicx}
\usepackage{caption}
\usepackage{subfigure}


\begin{document}

\begin{titlepage}
\begin{center}

        {\LARGE Computational Fluid Dynamics Modeling of Cathodic Arc Jet in Subsonic Flow Field  \par}
         \vskip 1em
        {\LARGE \selectlanguage{hebrew}מידול זרימה חישובית של סילוני גז מקשתות קתודיות בשדה זרימה תת-קולי \selectlanguage{english} \par}
        \vskip 2em
        A Master's Thesis Proposal \\
        {\tiny by} \\
        Anton Ronis\\
        \textit{ID} \texttt{314201005}\\
        \vskip 2em
        Primary Thesis Advisor \par
        Prof.~Igal Kronhaus\par
        {\Large \makebox[3in]{\hrulefill} \par}
        \vskip 1em
        {\small
            \today \\
            Faculty of Aerospace Engineering\\
            Technion - Israel Institute of Technology\\
            Technion City, Haifa 32000, Israel\\}
    \par

% prevent a page break from being put at the end of the title page so that 
            % the contents of the paper spill onto the title page
    % save the function of the \newpage macro so we can restore it later
    \global\let\newpagegood\newpage
    \global\let\newpage\relax
\end{center}
\end{titlepage}
% restore the \newpage command after creating the title page
\global\let\newpage\newpagegood

\section{Background}
The quest of reaching one's desired aerodynamic flow design parameters has been initiated since the beginning of human flight. One of the many paths which tries to accomplish this quest, is the concept of aerodynamic flow-control.
\par A flow-control device affects the original flow field, thus enabling controlling flow characteristics such as reduction of drag by delaying and/or eliminating separation zones. \par Several examples of flow-control devices include: bumps, vanes (vortex generators), synthetic jets, bleed injectors, plasma actuators and \textit{etc}.
\subsection*{Cathodic Arc induced Jets}
Recently \cite{1}, the characteristics of CAJs (Cathodic Arc Jets - gas flows generated by a cathodic arc, interacting with the molecules of the background gas) were experimentally measured in an atmospheric pressure environment.
\par This experiment is a continuation of a recent trend which deals with the application of plasma sources generated from a cathodic arc discharge for propulsion.
\par In \cite{1}, it is demonstrated that a CAJ pulse generated in an atmospheric environment, which is characterized by a time scale of hundreds of $\mu$s is capable of affecting the background still air by inducing a local flow field up to velocities of $\ge$ 100 $m/s$. 
\par The jet direction was shown to be controlled by the application of an external magnetic field which affected the direction's rotation according to Lorentz force direction. 
\par The results obtained in \cite{1} suggest the possible use of a CAJ as a flow-control device. 
\section{Objectives}
The proposed research contains two main objectives:
\begin{enumerate}
    \item The modeling of a CAJ in respect to a background flow field.
    \item Characterization of the effects of a CAJ on a 2D profile in a subsonic flow.
\end{enumerate}
The first objective deals with creating and evaluating an empirical CAJ model. The model should behave well with available data from \cite{1}. The model must deal with the flow properties of the CAJ, such as momentum, temperature, pressure and \textit{etc}.
\par The second objective deals with characterization of the flow properties of a CAJ placed on a 2D profile subjected to subsonic flow, by using the model obtained in the first part of the research. The study will focus on core flow properties aspects, mainly: 
\begin{enumerate}
   	\item \textbf{Drag}:\quad The effects of the CAJ on drag will be calculated. A comparison of the drag due to lift will be analyzed.
	\item \textbf{Stall Properties}:\quad Stall properties will be analyzed in the form of stall angle.
\end{enumerate}
\section{Methodology}
The research methodology is comprised of two aspects: creation of an empirical CAJ model and the application of this model in a flow field.
\subsection{CAJ Model}
A Cathodic-Arc-Jet empirical model will be constructed by data collected from \cite{1}. This data-driven model will be then evaluated by comparing the simulated gas jets with the data obtained from \cite{1}. The model will be realized as a numerical boundary condition in the flow solver.
\subsection{2D Flow Field}
A 2D set of Navier-Stokes equations (NS) containing the CAJ as a time lapsing boundary condition is solved.
\par A chosen RAS (Reynolds-Averaged Simulation) turbulent model which fits best to the evaluation test data will be used for the profile's flow properties study. Other models (\textit{Spalart-Allmaras}, $k-\omega$ \textit{etc.}) will be tested and accounted for by applying an uncertainty factor to the results. The tools used in the process are:
\begin{enumerate}
	\item \textbf{Mesh Generation}:\quad The numerical mesh will be generated by using \texttt{Pointwise}.
    \item \textbf{Flow Solver}:\quad The generated mesh will be numerically solved by using either a structured or unstructured grid solver (\texttt{openFOAM}/\texttt{SU$^2$}/\texttt{EZNSS}).
    \item \textbf{Flow Visualization}: The resultant flow field will be visualized by using flow visualization programs (\texttt{ParaView}/\texttt{Tecplot}) and \textsc{Matlab}.
\end{enumerate}

\begin{thebibliography}{99}

\bibitem{1}
  I. Kronhaus, S. Eichler and J. Schein,
  \emph{``Schlieren characterization of gas flows generated by cathodic arcs in atmospheric pressure environment”}. Applied Physics Letters 104, 063507 (2014); doi: 10.1063/1.4865397.

\end{thebibliography}

\end{document}
